\documentclass{llncs}

\title{A Correct Compiler for Continuation Marks}

\author{Kimball Germane \and Jay McCarthy}
\institute{Brigham Young University}

\begin{document}

\maketitle

\begin{abstract}

\end{abstract}

\section{Introduction}

\section{Related Work}

\section{Continuation Marks}

There are certain tools that are indispensable to some programmers that concern the
behavior of their programs: debuggers, profilers, steppers, etc. Without these tools,
these programmers cannot justify the adoption of a language, however compelling it might
otherwise be. Traditionally, these tools are developed at the same level the 
language is, privy to incidental implementation detail, precisely because that detail 
enables these tools to function. This is problematic for at least two reasons. First, 
it couples the implementation of the tool with the implementation of the language, which
increases the cost to port to other platforms. If users become dependent upon these tools,
it can stall the advancement of the language and the adoption of new language features.
Second and more critical, it makes these tools unsound. For instance, debuggers typically
examine programs which have been compiled without optimizations. In general, this means 
that the debugged program has different behavior than the deployed program. This is 
obviously undesirable.

It is desirable to implement such tools at the same level as the language, removing
dependency upon the implementation an instead relying on definitional and behavioral
invariants. Continuation marks are a language-level feature that provide the information
necessary for these tools to function. Furthermore, languages which require stack
inspection to enforce security policies (\emph{Java}, \emph{C\#}) or support aspect
oriented programming (\emph{aspectj}) can be defined in terms of a simpler language with
continuation marks \cite{clements2004tail}.

Continuation marks originated in PLT Scheme (now Racket \cite{plt-tr1}) as a stack 
inspection mechanism. In fact, the \emph{Java} and \emph{C\#} languages rely on a similar 
stack inspection to enforce security policies of which continuation marks can be seen as 
a generalization. Surprisingly, continuation marks can be encoded in any language with 
exception facilities \cite{pettyjohn2005continuations} which fact has led to their 
experimental addition to Javascript \cite{clements2008implementing}.

\section{$\lambda_{cm}$}

\section{Transformation}

\section{Testing}

\section{Proof}

\section{Future Work}

\section{Conclusion}

\end{document}
